\chapter*{Abstract}

Webmasters can increase the visibility of a web page by adding machine-readable data to the page. This allows applications like search engines to understand the content of the web page and increase search visibility which can indirectly impact a web page's search ranking. 

Schema.org provides a shared vocabulary for webmasters to use in order to create the machine-readable data markup. However, due to the large number of terms in Schema.org's vocabulary webmasters find difficulty in choosing which terms to use. Schema.org's vocabulary also has limited coverage and does not provide terms in every area for example life sciences, requiring extensions to fill in the missing gaps. Bioschemas proposes new types of information from life sciences to be added to Schema.org's core vocabulary to better support biological types not included in Schema.org. Additionally, it provides a profile layer that enhances the Schema.org model by recommending the most relevant terms,
to help webmasters in creating their machine-readable data markup.

Over the course of this project, I have developed a system that allows users to generate markups based on Bioschemas.org profiles to increase the visibility of data from the life science community. Through an initial usability evaluation, members of the life science community found the prototype system to be appealing, easy to use and navigate but fed back that more information needed to be provided to the user. A further usability evaluation was carried out by comparing the final system against a second similar system. Results showed that all test subjects preferred the final system and on average were able to complete the tasks faster compared to the second similar system.
